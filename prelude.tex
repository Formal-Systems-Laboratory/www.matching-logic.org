\usepackage[T1]{fontenc}
\usepackage{mathtools}

\usepackage[inline]{enumitem}
\usepackage{bigdelim}
\usepackage{multirow}

\usepackage{subcaption}

\usepackage{tabularx}

\usepackage{tikz}
\usetikzlibrary{automata, positioning, arrows}
\tikzset{
->, % makes the edges directed
>=stealth', % makes the arrow heads bold
node distance=3cm, % specifies the minimum distance between two nodes. Change if necessary.
intermediate/.style={minimum height=1em,align=center}, % sets the properties for each ’state’ node
state/.style={rectangle,rounded corners,minimum height=3em,align=center,fill=gray!05,draw}, % sets the properties for each ’state’ node
task/.style={rectangle,align=center,draw}, % sets the properties for each ’state’ node
tool/.style={rectangle,align=center,draw}, % sets the properties for each ’state’ node
initial text=$ $, % sets the text that appears on the start arrow
}



% Set theory


% Logic

% Matching Logic



% Regular expressions








% For the presentation
\usepackage{forest}
\forestset{
  rule/.style 2 args={
      edge label={node[midway, right] {\prule{{#1}}}},
      edge+={->}
        },
}

\setcounter{tocdepth}{1}


\usepackage{listings}


\lstdefinelanguage{mm0}{
morekeywords={theorem,Pattern},
literate=
*{=>}{{$\Rightarrow$}}1
 {<->}{{$\liff$}}1
 {\\/}{{$\lor$}}1
 {/\\}{{$\land$}}1
 {Alpha}{{$\alpha$}}1
 {Beta}{{$\beta$}}1
 {epsilon}{{$\epsilon$}}1
 {derivative}{{$\delta$}}1
 {...}{{$\ldots$}}2
}

\usepackage{longtable}

\newtheorem{theorem}{Theorem}[section]
\newtheorem{definition}{Definition}[section]
\newtheorem{lemma}{Lemma}[section]

